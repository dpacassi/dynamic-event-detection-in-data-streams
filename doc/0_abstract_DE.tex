\section*{Zusammenfassung}
\label{sec:0_zusammenfassung}

Die Erkennung von Ereignissen in Datenströmen kann sich als schwierig erweisen, vor allem dann,
wenn sich die Definition, Inhalte oder Eigenschaften eines Ereignisses im Laufe der Zeit verändern kann.

Diese Bachelorarbeit fokussiert auf die Entwicklung und Evaluation einer Online-Clustering Lösung,
in welcher Ereignisse entweder als Veränderungen bestehender Cluster oder aber als Bildung neuer Cluster
definiert sind.
Die Lösung ist eine Text-Mining Software, welche über einen Datenstrom neue News-Artikel erhält
und diese verarbeitet.
Dabei werden Artikel aufgrund ihrer Ähnlichkeit zu anderen Artikeln verschiedenen Clustern zugewiesen.
Die Annahme ist, dass sehr ähnliche Artikel über dasselbe Thema schreiben.
Zusätzlich wurde für die Evaluation der Clusteringalgorithmen eine eigene Bewertungsfunktion entwickelt.

Im ersten Teil dieser Arbeit wurde nach einem geeigneten Datensatz gesucht,
in welchem Inhalte gemäss Ähnlichkeit gruppiert werden.
Die umgesetzte Lösung verwendet HDBSCAN als Clustering Methode und vergleicht diese
mit dem State-of-the Art Verfahren \textit{k}-means.
Dabei stellte sich heraus, dass die Verwendung von HDBSCAN Vorteile bei der Performanz,
wie aber auch bei der Präzision gegenüber \textit{k}-means aufweist.
Des Weiteren wurden auch verschiedene Textvorverarbeitungsmethoden evaluiert.
Der Einsatz von Text Lemmatisierung und des Tf-idf-Masses verbesserten
das Clustering in Hinsicht auf Präzision.
Bei der abschliessenden Evaluation stellte man fest,
dass nicht zugewiesene News-Artikel die Präzision des Clusteringverfahrens reduzieren.

Die resultierende Präzision des Clusteringverfahrens beträgt 72\% bei einer Standardabweichung von 12\%.
Die Präzision zur Erkennung neuer Ereignisse beträgt 62\% bei einer Standardabweichung von 43\%.
Die Erkennung von Änderungen bestehender Ereignisse ergibt eine
Präzision von 69\% bei einer Standardabweichung von 16\%.
Eine Fortsetzung dieser Arbeit sollte die Verbesserung des Clusterings sein,
um die Präzision bei der Erkennung von Ereignissen zu erhöhen.
