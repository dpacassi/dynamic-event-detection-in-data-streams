% Der Umfang des Abstracts sollte nicht mehr als eine A4-Seite sein
% (nach DIN / ISO / ANSI beschränkt sich ein Abstract auf ca. 250 Wörter, also eine halbe A4-Seite).

\section*{Abstract}

Die Erkennung von Ereignissen in Datenströmen kann sich als schwierig erweisen, vorallem dann,
wenn sich die Definition eines Ereignisses im Laufe der Zeit verändern kann.

Diese Arbeit fokussiert sich auf die Entwicklung und Evaluation einer Online-Clustering Lösung,
in welcher Ereignisse entweder als Cluster-Veränderungen oder aber als Bildung neuer Cluster
verstanden werden.
Die Lösung ist eine Text-Mining Software, welche über einen Datenstrom neue Newsartikel erhält
und diese verarbeitet.
Dabei werden Artikel aufgrund ihrer Ähnlichkeit zu anderen Artikeln verschiedenen Clustern zugewiesen.
Die Annahme ist, dass sehr änhliche Artikel über dasselbe Thema schreiben.
Ausserdem wurde für die Qualitätsprüfung des Clusterings eine eigene Bewertungsfunktion entwickelt.

Im ersten Teil dieser Arbeit wurde nach einem geeigneten Datensatz gesucht,
in welcher Inhalte gemäss Ähnlichkeit gruppiert wurden.
Die umgesetzte Lösung verwendet HDBSCAN als Clustering Methode und vergleicht diese mit
dem bekannten \textit{k}-means Verfahren.
Dabei stellte sich heraus, dass HDBSCAN nicht nur schneller, aber auch präziser als \textit{k}-means ist.
Desweiteren wurden auch verschiedene Textvorverarbeitungsmethoden evaluiert.
Der Einsatz von Text Lemmatisierung und des Tf-idf-Masses erwiesen sich hierbei als äusserst erfolgsversprechend.

Bei der Prüfung der Lösung in einem simulierten Testlauf, führten Ungenauigkeiten im gesamten Clustering
zu einem negativen Einfluss auf die Ereigniserkennung.
Dies resultierte in einer signifikanten Fehlerquote bei der Erkennung neuer Ereignisse,
während die Erkennung von Änderungen bestehender Ereignisse bessere Ergebnisse erzielte.
Insgesamt wird die Fehlerrate bei der Ereigniserkennung als zu hoch für reale Anwendungen angesehen,
und eine mögliche Fortsetzung dieser Arbeit könnte die Verbesserung des Clusterings sein,
um die Fehlerrate bei der Erkennung von Ereignissen zu minimieren.
