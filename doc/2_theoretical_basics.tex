\section{Theoretical basics}

\subsection{Data Representation}

\subsubsection{Word Count}

\subsubsection{Tfidf}

\subsection{Clustering}
Clustering finds similarities in different news articles based on their content and groups them together, while unrelated news are regarded as noise. The challenge now araises to find an appropriate clustering method, which is able to work with data of varying densities and of high dimensionality.

TODO why hdbscan

\subsubsection{\textit{k}-means clustering}
KMeans is a centroid-based clustering algorithm.

TODO explain some more

\subsubsection{HDBSCAN}
HDBSCAN is a hierarchical density-based clustering algorithm \cite{McInnes2017}. It extends the well known [insert citation] DBSCAN algorithm and reduces its sensitivity for clusters of varying densities. Another important quality of HDBSCAN is, that it does not need to know the number of clusters up front.

TODO explain some more
