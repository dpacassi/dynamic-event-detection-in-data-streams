% Der Umfang des Abstracts sollte nicht mehr als eine A4-Seite sein
% (nach DIN / ISO / ANSI beschränkt sich ein Abstract auf ca. 250 Wörter, also eine halbe A4-Seite).

\section*{Abstract}

% Einleitung: definiert die Problematik und begründet die Relevanz der Arbeit.

% While searching for events in a data stream, the definition of an event is not always clear. Providing static definitions, as most approaches do,
% does not function well for dynamic data streams, which may change the definition of an event over time.

Event detection in dynamic data streams can be challenging, 
especially if the definition or the properties of an event change over time.

% Solution
This thesis focuses on developing and evaluating a methodology based on online clustering, where events can be considered either as changes in clusters over time or as the creation of new clusters. 
The methodology will be applied in the domain of text mining, 
with data streams consisting of incoming news articles. 
This allows news articles to be clustered based on their similarity, 
as similar news articles are considered to be about the same news story. 
In addition, the evaluation of the clustering quality is measured with a custom scoring function.

% Summary of results
The first part of this work is in determining a suitable data set, 
which will be the subject of the clustering and provide the ground truth for evaluating the results. 
The evaluation focuses on HDBSCAN as the clustering method and compares it with the state-of-the-art \textit{k}-means, 
where HDBSCAN is both faster and more precise. 
Moreover, different text preprocessing methods and vector space models are evaluated, with text lemmatisation and tf-idf providing the most promising results. 
Once applied in a simulated online setting we found the noise rate in the overall clustering to reduce the precision in the event detection.

The resulting precision of the clustering is 72 \% with a standard deviation of 12 \%. 
The precision for detecting new events results in 62 \% with a standard deviation of 43 \%, 
while the detection of changes in existing events results in a precision of 69 \% with a standard deviation of 16 \%. 
A continuation of this work should focus on improving the overall clustering to increase the precision of the event detection.