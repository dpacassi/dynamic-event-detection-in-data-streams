% Der Umfang des Abstracts sollte nicht mehr als eine A4-Seite sein
% (nach DIN / ISO / ANSI beschränkt sich ein Abstract auf ca. 250 Wörter, also eine halbe A4-Seite).

\section*{Abstract}

% Einleitung: definiert die Problematik und begründet die Relevanz der Arbeit.

While searching for events in a data stream, the definition of an event is not always clear. Providing static definitions, as most approaches do,
does not function well for dynamic data streams, which change over time.


% Solution
This thesis focuses on developing and evaluating a methodology based on online clustering, where events can be considered either as changes in clusters over time or as the creation of new clusters. The methodology will be applied in the domain of text mining, with data streams consisting of incoming news articles. This allows news articles to be clustered based on their similarity, as similar news articles are considered to be about the same news story. In addition, the evaluation of the clustering quality is measured with a custom scoring function.

% Summary of results
The first part of this work is in determining a suitable data set, which will be the subject of the clustering and at the same time provide the ground truth for evaluating the results. The evaluation focuses on HDBSCAN as the clustering method and compares it with \textit{k}-means, where HDBSCAN is both faster and more accurate. Moreover, different text preprocessing methods and vector space models are evaluated, with text lemmatisation and tf-idf providing the most promising results. Once applied into an simulated online setting, inaccuracies in the overall clustering have a larger impact on event detection. This results in a significant error rate in the detection of new events, while the detection in changes of existing events shows better results. Overall the error rate in event detection is considered to be too high for real world applications and a possible continuation of this work could be improving the clustering to decrease the error rate in detecting events.


