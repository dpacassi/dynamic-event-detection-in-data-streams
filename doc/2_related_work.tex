\section{Related Work}
\label{sec:2_related_work}

Text based event detection is a diverse field with an increasing amount of available information online.
With the popularity of social media, a lot of research around event detection
has been done on micro blogs\cite{microblog_clustering}, such as Twitter\cite{twitter_survey, social_media_survey}.

\paragraph{Text Preprocessing}
% \label{subsec:2_text_preprocessing}
Text Stemming and Text Lemmatization have been around for decades and both are proven to not only work
in theory but also in practice.
So it comes as no surprise that many people try to improve them with new methods such as
Ultra-stemming\cite{UltraStemming} or corpus-based stemming algorithms\cite{CorpusBasedStemming}.

There are also different papers comparing the preprocessing methods
with each other\cite{Suryanarayana2015SteppingTA, BounabiMS17}.

With that many resources available, it is important to be aware that the different methods
score differently for each language and each data set.
Having enough time for evaluating potential candidates is crucial.

\paragraph{Clustering}
% \label{subsec:2_clustering}
This thesis focuses on news articles as the primary data source.
Text based clustering as a technique for event detection
has already been explored with different approaches such as
using custom methods based on neural networks\cite{text_clustering_topic_detection}
or by using a modified version of DBSCAN
to account for its sensitivity for differences in cluster densities\cite{dbscan_martingale}.

Based on the promising results with DBSCAN, we want to further explore text clustering
using its successor HDBSCAN\cite{McInnes2017} and apply it in an online setting.
Regarding the clustering validation, there has already been research into recognizing biases
of different scoring functions\cite{Wu:2009:ARM:1557019.1557115}
and developing custom scoring functions as a result\cite{gates2017comparing}.
