\section{Abstract}

How can events be recognized in a data stream? When searching for new events in data streams,
there is always the problem of what exactly an event is.
Once this first question has been clarified, however, further questions arise.
Usually it is not enough to define an event statically, but the definition develops dynamically over time.
However, most procedures for event recognition are set statically when the system is started.
Furthermore, the dynamics of the stream must always be taken into account,
otherwise blockages and overflow in the system can occur.

The aim of this thesis is to develop a methodology how the mentioned aspects can be implemented on suitable data streams.

The tasks of this thesis are as following:
\begin{itemize}
  \item Literature research on the topic of event recognition
  \item Learning tools like Spark DataFrames, TensorFrames and Spark Streaming
  \item Conception of a use case and an approach to event recognition
  \item Evaluation of the results
\end{itemize}
