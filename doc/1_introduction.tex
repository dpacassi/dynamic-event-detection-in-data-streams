\section{Introduction}
\label{sec:1_introduction}

\subsection{Problem Formulation}
\label{subsec:1_problem_formulation}

While searching for events in a data stream, the definition of an event is not always given.
Providing static definitions, as most approaches do,
does not suffice for dynamic data streams, which change over time.
Additionally, the behaviour of the data stream is an important factor in itself,
since blockages of overflows in the system have to be prevented.

An example for a dynamic data stream can be found in a stream of news articles,
which are published in irregular time intervals and different quantities over time.
Detecting events based on an incoming stream of news articles is therefore a challenging task.

The goal is to develop and evaluate a methodology to detect events in a dynamic data stream of news articles.

\subsection{Motivation}
\label{subsec:1_motivation}

Today's environment is rapidly changing.
Most news articles are not spread via published media anymore but digital.
With the overwhelming amount of news, finding relevant news is difficult.

As reading all news articles is no option, an automated methodology becomes necessary.
This is where our work becomes relevant.
We want to detect events from a news data stream, in particular identifying new headlines.

Since our solution is based on text data, any data in text form is applicable.
With technologies such as speech recognition, the data could also initially be acoustic
and converted to text before being entered into our application.

This would open up use cases with smart speakers such as \textit{Amazon Echo} or \textit{Google Assistant}.
