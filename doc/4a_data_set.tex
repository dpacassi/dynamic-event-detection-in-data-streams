\subsection{Data Set}
\label{subsec:4a_data_set}

Before any clustering method can be implemented or evaluated,
it is important to rely on the right data set for training and evaluation.

\subsubsection{Data Set Candidates}
\label{subsubsec:4a_data_set_candidates}

As our goal is to detect events in data streams, we have evaluated different data sets and
their possibilities to extract events from their data themselves.
\tabref{tab:data_set_candidates} lists the evaluated data set candidates.

\begin{table}[h]
    \centering
    \begin{tabular}{|l|r|l|}
    \hline
    \textbf{Data set} & \textbf{Number of rows} & \textbf{Description} \\ \hline
    GDELT 2.0\cite{GDELT} & over 575,000,000 & Print and web news from around the world. \\ \hline
    ChallengeNetwork\cite{ChallengeNetwork} & 4,449,294 & Network packages including anomalies. \\ \hline
    One Million Posts Corpus\cite{OneMillionPostsCorpus} & 1,011,773 & User comments to news articles. \\ \hline
    Online Retail Data Set\cite{OnlineRetailDataSet} & 541,909 & Customer retail purchases of one year. \\ \hline
    News Aggregator Data Set\cite{NewsAggregatorDataSet} & 422,937 & Clustered news articles. \\ \hline
    Dodgers Loop Sensor Data Set\cite{DodgersLoopSensorDataSet} & 50,400 & Number of cars driven through a ramp. \\ \hline
    10k German News Articles\cite{10kGNAD} & 10,273 & German news articles. \\ \hline
    \end{tabular}
    \caption{Evaluated data set candidates ordered by data set size.}
    \label{tab:data_set_candidates}
\end{table}

We could extract events from all data sets mentioned in \tabref{tab:data_set_candidates}.\\
The extracted events could be as follows:

\begin{itemize}
    \item Network packages
        \begin{itemize}
            \item Cyber attacks depending on suspicious packets.
        \end{itemize}
    \item User comments
        \begin{itemize}
            \item Change of public opinion during time.
        \end{itemize}
    \item Retail purchases
        \begin{itemize}
            \item Change of purchasing behavior based on product choices.
        \end{itemize}
    \item Traffic
        \begin{itemize}
            \item Traffic changes due to baseball games.
        \end{itemize}
    \item News articles
        \begin{itemize}
            \item Development of a certain news story.
        \end{itemize}
\end{itemize}

However, from above data sets only two contained prelabeled clusters:

\begin{enumerate}
    \item Dodgers Loop Sensor Data Set
        \begin{itemize}
            \item 50,400 sensor activities belonging to 81 games.
        \end{itemize}
    \item News Aggregator Data Set
        \begin{itemize}
            \item 422,937 news articles belonging to 7,231 news stories.
        \end{itemize}
\end{enumerate}

As we did not want to lose too much time in manually labeling data, we decided to go with one of these two.
Regarding our two options, our choice was simple:\\
We went for the \textbf{The News Aggregator Data Set} since it not only provided more data,
but our work built on the news articles use case could later be continued with real live data.
The \textbf{GDELT 2.0} data set for example, provides around 1,000 to 2,000 new news articles every 15 minutes.

\subsubsection{Data Retrieval}
\label{subsubsec:4a_data_retrieval}

Unfortunately the data set did not contain the news articles themselves but rather only the URL's to the news articles.
This was done so due to copyright restrictions on the content.
Fortunately there are web scraping tools designed to retrieve the content from news portals specifically.
We decided to use Newspaper3k\cite{newspaper3k},
a Python3 library that allows us to retrieve the news article texts from the news portals easily.

The library only requires an URL to download and extract the news article from a website,
see \lstref{lst:newspaper3k_code}.

\begin{lstlisting}[language=Python, caption=Retrieve the news article from an URL., label={lst:newspaper3k_code}]
    from newspaper import Article

    url = 'http://fox13now.com/2013/12/30/new-year-new-laws-obamacare-pot-guns-and-drones/'
    article = Article(url)
    article.download()
    article.text # Contains the article's text.
\end{lstlisting}

All we had to do now, is to run this code for all news articles.
To speed this process up, we loaded the data set into a database and ran 8 concurrent processes which
retrieved the news articles content from the web portals in different batches.

\subsubsection{Data Cleansing}
\label{subsubsec:4a_data_cleansing}

The data set contains news articles collected from March 10th to August 10th 2014.
Five years later, many resources are not online anymore or are not accessible from Europe due to Europe's \Gls{gdpr}.
We have used the SQL query in \lstref{lst:valid_news_articles_sql} to filter out news articles that were most likely corrupt.

\begin{lstlisting}[language=SQL, caption=Retrieve valid news articles., label={lst:valid_news_articles_sql}]
    SELECT *
    FROM news_article
    WHERE
        newspaper_text IS NOT NULL
        AND TRIM(COALESCE(newspaper_text, '')) != ''
        AND hostname NOT IN ('newsledge.com', 'www.newsledge.com')
        AND newspaper_text NOT LIKE '%GDPR%'
        AND newspaper_text NOT LIKE '%javascript%'
        AND newspaper_text NOT LIKE '%404%'
        AND newspaper_text NOT LIKE '%cookie%'
        AND newspaper_keywords NOT LIKE '%GDPR%'
        AND newspaper_keywords NOT LIKE '%javascript%'
        AND newspaper_keywords NOT LIKE '%404%'
        AND newspaper_keywords NOT LIKE '%cookie%'
        AND title_keywords_intersection = 1
\end{lstlisting}

From the original 422,937 news articles, 235,070 were still accessible to us.
They belong to 7'183 different news stories.

\subsubsection{News Article Example}
\label{subsubsec:4a_news_article_example}
It is important to check and verify the data before working with it.
See following news article we have scraped from thelocal.ch\cite{HRGiger} for example:

\begin{quote}
    The surrealist painter, sculptor and set designer, known as H.R. Giger,
    died in hospital following a fall, according to the report.

    A native of the canton of Graubünden, he was best known for his design of Alien,
    an American science-fiction horror film from 1979 directed by Ridley Scott,
    and was part of the special effects team that won an Academy Award the following year.

    Giger’s design for the film was inspired by his painting Necronom IV.

    He was known for his airbrushed silver and grey canvasses
    depicting nightmarish dreamscapes and “biomechanical” human figures linked to machines.

    Born in Chur in 1940, he studied architecture and industrial art at
    Zurich’s University of Applied Sciences (ZHAW).

    A friend of Timothy Leary who was influcenced by surrealist painter Salvador Dali,
    he reportedly suffered from nightmares that he said inspired some of his work.

    Books of his paintings and use of his art for music albums and in publications,
    such as the American science and science fiction Omni, led to his rise in international fame.

    His artwork adorned the cover of albums by artists such as Emerson Lake \& Palmer,
    the Dead Kennedys and French singer Mylène Farmer.

    In addition, to Alien, Giger was also involved in films such as Poltergeist II (1986),
    Alien III (1992) and Species (1995).

    In 1998, Giger acquired the Château St. Germain in Gruyères,
    a village in the canton of Fribourg, to house his work in the H.R. Giger Museum.

    He lived and worked in Zurich-Seebach, SRF said.

    For all his fame, Giger said he did not gain "big money" from his work in the movies,
    Zurich newspaper Tages Anzeiger reported.

    Others, he said, cashed in on his creative work.

    "My design was done and changed," Giger is quoted as having said.
    "The film business is a gangster business."
\end{quote}

The news article was scraped successfully without any data noise that does not belong to the article.

\paragraph{Text Stemming}
The stemmed version of the text is:
\begin{quote}
    the surrealist painter, sculptor set designer, known h r giger, die hospit follow fall,
    accord report a nativ canton graubünden, best known design alien,
    american science-fict horror film 1979 direct ridley scott,
    part special effect team academi award follow year giger design film inspir paint necronom
    iv he known airbrush silver grey canvass depict nightmarish dreamscap “biomechanical”
    human figur link machin born chur 1940, studi architectur industri art zurich univers appli scienc (zhaw)
    a friend timothi leari influcenc surrealist painter salvador dali, report suffer nightmar said
    inspir work book paint use art music album publications, american scienc scienc fiction omni,
    led rise intern fame his artwork adorn cover album artist emerson lake \& palmer, dead kennedi
    french singer mylèn farmer in addition, alien, giger also involv film poltergeist ii (1986),
    alien iii (1992) speci (1995) in 1998, giger acquir château st germain gruyères, villag canton fribourg,
    hous work h r giger museum he live work zurich-seebach, srf said for fame, giger said gain "big money"
    work movies, zurich newspap tage anzeig report others, said, cash creativ work "mi design done changed,"
    giger quot said "the film busi gangster busi "
\end{quote}

\paragraph{Text Lemmatization}
The lemmatized version of the text is:
\begin{quote}
    the surrealist painter , sculptor and set designer , know as h.r.giger , die in hospital follow a fall ,
    accord to the report. a native of the canton of graubünden , he was best known for his design of alien ,
    an american science - fiction horror film from 1979 direct by ridley scott ,
    and was part of the special effect team that win an academy award the following year.
    giger ’s design for the film was inspire by his painting necronom iv.
    he was know for his airbrushed silver and grey canvass depict nightmarish dreamscape
    and " biomechanical " human figure link to machine. bear in chur in 1940 ,
    he study architecture and industrial art at zurich ’s university of applied sciences ( zhaw ).
    a friend of timothy leary who was influcence by surrealist painter salvador dali ,
    he reportedly suffer from nightmare that he say inspire some of his work.
    book of his painting and use of his art for music album and in publication ,
    such as the american science and science fiction omni , lead to his rise in international fame.
    his artwork adorn the cover of album by artist such as emerson lake \& palmer ,
    the dead kennedys and french singer mylène farmer.
    in addition , to alien , giger was also involve in film such as poltergeist ii ( 1986 )
    , alien iii ( 1992 ) and species ( 1995 ).in 1998 , giger acquire the château st.germain in gruyères ,
    a village in the canton of fribourg , to house his work in the h.r.giger museum.
    he live and work in zurich - seebach , srf say. for all his fame ,
    giger say he did not gain " big money " from his work in the movie ,
    zurich newspaper tages anzeiger report.others , he say , cash in on his creative work."
    my design was done and change , " giger is quote as having say."
    the film business is a gangster business."
\end{quote}

\paragraph{Keyphrase Extraction}
The extracted Keyphrases from this article are:
\begin{quote}
    Zurich newspaper Tages Anzeiger, french singer Mylène Farmer, surrealist painter Salvador Dali,
    painting Necronom IV, Zurich ’s University, special effect team, american science, science fiction Omni,
    fiction horror film, H.R. Giger, Giger, Château St. Germain, H.R. Giger Museum, Ridley Scott,
    following year, human figure, nightmarish dreamscape, Academy Award, canton, grey canvass, Applied Sciences
    industrial art, Timothy Leary, airbrushed silver, design, Alien, music album, international fame,
    Dead Kennedys, Emerson Lake, Alien III, Poltergeist II, film, Graubünden, work, known, native, fall, report,
    hospital, Chur, architecture, Born,biomechanical, machine, big money, friend, ZHAW, album, fame, Books,
    nightmare, publication, artist, cover, artwork, Palmer, rise, addition, creative work, specie, Fribourg,
    village, Gruyères, SRF, Seebach, designer, movie, film business, gangster business, sculptor
\end{quote}

\paragraph{Named Entity Recognition}
The recognized named entities are:
\begin{quote}
    H.R., Alien, Ridley, Necronom, Zurich, ZHAW, Timothy, Salvador, Omni, Emerson, Dead, Mylène, Alien, Alien,
    Species, Gruyères, H.R., SRF, Giger, Tages
\end{quote}

\paragraph{Conclusion}
Every text preprocessing method worked as expected.
Text Lemmatization correctly converted the terms to their dictionary base roots
which results in quite readable text.
As Text Stemming simply cuts off term endings based on a static rule set, the result is less
good readable than the Lemmatized text.
Keyphrase Extraction was able to extract 71 terms, while only 20 Named Entities were recognized.
